\documentclass[11pt]{article}
\usepackage{multicol}
\setlength{\columnseprule}{1pt} % separation line between columns
\setlength{\parindent}{0pt} % paragraph indentation

\usepackage[top=2cm, bottom=2cm, left=2cm, right=2cm]{geometry}
\usepackage[T1]{fontenc}
\usepackage[utf8]{inputenc}
\usepackage[francais]{babel}
\usepackage{textcomp}

\usepackage{hyperref}
\hypersetup{
	colorlinks=true,       	% false: boxed links; true: colored links
	linkcolor=black,          	% color of internal links
	urlcolor=blue,           	% color of external links
	citecolor=blue
}

\usepackage{dashrule}
\usepackage{wrapfig}
\usepackage{graphicx}
\usepackage{enumitem}
\usepackage{wrapfig}
\usepackage{cancel} % diagonal strikeout
\usepackage[margin=1cm]{caption}
\setdescription{leftmargin=1cm,labelindent=0.5cm}

\usepackage{amsmath}
\usepackage{amssymb}
\usepackage{amsfonts}

\newcommand\mathd[0]{\mathrm{d}} 

\usepackage{blindtext}

% Colors
\usepackage[usenames,dvipsnames]{xcolor}
\definecolor{session_bg}{RGB}{25,25,25}
\definecolor{grey2}{rgb}{0.3,0.3,0.3}
\definecolor{dkgreen}{rgb}{0,0.6,0}
\definecolor{gray}{rgb}{0.5,0.5,0.5}
\definecolor{mauve}{rgb}{0.58,0,0.82}
\definecolor{blue}{rgb}{0,0,0.7}

% Colored frame
\usepackage{framed}
\definecolor{shadecolor}{rgb}{0.96,0.96,0.96}
\definecolor{TFFrameColor}{rgb}{0.96,0.96,0.96}
\definecolor{TFTitleColor}{rgb}{0.00,0.00,0.00}

% Redefine leftbar environment
\newlength{\leftbarwidth}
\setlength{\leftbarwidth}{1pt}
\newlength{\leftbarsep}
\setlength{\leftbarsep}{10pt}

\newcommand*{\leftbarcolorcmd}{\color{leftbarcolor}} % as a command to be more flexible
\colorlet{leftbarcolor}{gray}

\renewenvironment{leftbar}{%
    \def\FrameCommand{{\leftbarcolorcmd{\vrule width \leftbarwidth\relax\hspace {\leftbarsep}}}}%
    \MakeFramed {\advance \hsize -\width \FrameRestore }%
}{%
    \endMakeFramed
}

\usepackage{listings}
\lstloadlanguages{C,sh}
\lstdefinestyle{session}{
	numbers=left,	
	tabsize=4,
	frame=single, % cadre autour du code
	basicstyle=\small\ttfamily\color{white},
	numberstyle=\scriptsize\ttfamily,
	backgroundcolor=\color{session_bg},
	showstringspaces=false,
	keywordstyle=\color{OliveGreen},
	stringstyle=\color{BrickRed},
	commentstyle=\color{grey2}\it,
	stepnumber=1
}
\lstdefinestyle{C}{
    language=[ANSI]C,
	basicstyle=\scriptsize,
	numbers=left,                   % where to put the line-numbers
  	numberstyle=\tiny\color{gray},
	commentstyle=\color{dkgreen},
	frame=single,                   % adds a frame around the code
 	rulecolor=\color{black},
	emph={},
	emphstyle=\color{mauve},
	morekeywords={},
	keywordstyle={\color{blue}},
	showstringspaces=false
}

% Title page
\title{Création d'une interface graphique et interactive \\ pour jeux de rôles papier}
\author{
	Damien Martel \\ \href{mailto:damien.martel@edue.esiee.fr}{damien.martel@edu.esiee.fr} \and
	Bertrand Le Mée \\ \href{mailto:bertrand.lemee@edu.esiee.fr}{bertrand.lemee@edu.esiee.fr} \and
	Vincent Lindivat \\ \href{mailto:vincent.lindivat@edu.esiee.fr}{vincent.lindivat@edu.esiee.fr} \and
	Frédéric Nguyen \\ \href{mailto:frederic.nguyen@edu.esiee.fr}{frederic.nguyen@edu.esiee.fr}
}
\date{\today}

\begin{document}
\maketitle
\newpage
\hfill
\newpage
\tableofcontents
\newpage

\section{Introduction}

\section{Méthode agile SCRUM}

A l'occasion de ce projet, nous avons pu découvrir la méthode de développement
 agile SCRUM  et dans le même mouvement, appliquer cette méthode à la réalisation de notre projet.\\
 
Dans le cadre SCRUM, les trois entités suivantes se rassemblent autour du projet :

\begin{description}
	\item[Product Owner] \hfill \\
		Il définit les fonctionnalités du projet et décide de valider ou non les résultats. Il met en place les dates à respecter. Le Product Owner établit aussi le \textit{Backlog} avec le Scrum Master et l'équipe, mais il peut modifier un \textit{Sprint} si besoin.
	\item[Scrum Master] \hfill \\
		Il s'occupe de l'équipe de développement en veillant à ce la méthode SCRUM soit bien appliquée. Il conseille son équipe et s'assure qu'elle progresse correctement. Il gère aussi les relations extérieures, particulièrement avec le Product Owner.
	\item[Equipe de développement] \hfill \\
		Généralement constituée de 5 à 10 personnes, l'équipe regroupe tout type de rôle. Une équipe travaille sur un \textit{Sprint} et est libre de s'organiser d'elle-même.
\end{description}

La progression du projet est basée sur un \textbf{Backlog}, faite par le Product Owner, le Scrum Master et l'équipe, qui est une liste ordonnée par priorité et complexité de temps de tâches à réaliser. Les priorités sont choisies par le Product Owner et peuvent être revues selon le courant des événements. \\

Durant le projet, le temps est divisé en séquence ayant une durée moyenne de 2 semaines, ces séquences sont appelées \textbf{Sprint}. Avant chaque Sprint, l'équipe choisit dans le Backlog l'ensemble des tâches qu'elle compte pouvoir réaliser. \\

Quotidiennement, une réunion d'une quinzaine de minutes \textbf{debout} se fait dans l'équipe afin de faire le point sur ce qui a été fait le jour précédent et ce qui va être fait le jour de la réunion. Lorsqu'un Sprint est terminé, le Product Owner, le Scrum Master et l'équipe se rassemble pendant 30 minutes pour faire le point ce qui a marché et ce qui n'a pas marché afin de mieux préparer et agir lors du prochain Sprint. \\

N'étant qu'une équipe de 4 et le projet ne durant que 2 mois, nous avons dû réadapter la méthode. Nous nous sommes convenus de faire des Sprints de une semaine et travailler de façon très souple pendant les temps de travail.

\section{Manuel d'utilisation}

\section{Conclusion}

\end{document}