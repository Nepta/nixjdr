\subsection{Qt}

Qt est une librairie multi-plateforme codée en C++ qui permet de créer des applications graphiques. \\

Qt se base sur des QObject pour représenter tout élément d'une application et est héritée par toutes les autres classes existantes de la librairie. Pour une interface utilisateur, il existe une classe QWidget qui permet de représenter tout objet qu'on peut placer dans une fenêtre.\\

Pour tous les objets, il est possible de les faire interagir entre eux grâce à un système de signaux et slots que propose Qt et qui remplace l'utilisation du patron de conception observateur. Nous pouvons alors connecter deux QObject entre-eux, où le signal d'un objet correspond à un slot de l'autre objet. Les signaux et slots sont des méthodes qui peuvent être ajoutées dans les classes héritant de QObject, sachant qu'un signal est une méthode ne possédant pas de corps. Durant le programme, lorsqu'un signal est émis, les slots qui sont connectés dessus s'exécutent. \\

Qt possède également un environnement de développement qui est composé de deux outils : Qt Creator et Qt Designer. Qt Creator est l'IDE dans lequel nous éditons notre code et possède un compilateur et un débugger pour aider au développement. Qt Designer permet de créer graphiquement des interfaces utilisateurs et génère des fichiers sources que nous pouvons utiliser dans le code. \\

Qt est donc une librairie très puissante et un framework efficace pour développer des projets en C++ sous plusieurs systèmes d'exploitation.