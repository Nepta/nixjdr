\section{Introduction}

Le jeu de rôle (JdR) papier permet à un groupe de joueurs, réunis dans la même salle, d'incarner des personnages dans un univers fictif. Un Maître du jeu (MJ) les guide dans cette univers, leur proposant problèmes, quêtes et discussions avec d'autres personnages fictifs, et orchestrant l'ensemble de la partie, ou "session". Afin de rendre l'univers plus immersif, de nombreux supports physiques sont utilisés: cartes, dés, plateaux, jetons, dessins...\\
Cependant, il n'est pas toujours possible de réunir l'ensemble du groupe pour une session, selon les emplois du temps personnels. Cela mène à des entorses scénaristiques, nottament lors de l'abence d'un personnage possédant des informations clés pour l'aventure.\\

Ayant ces éléments à l'esprit, nous avons développé une interface graphique qui permet d'organiser des parties de jeu de rôle à distance sur réseau. Nous avons fait attention à ne pas dépayser les utilisateurs en conservant au mieux l'atmosphère du jeu de plateau habituel.\\

Ce projet est une opportunité pour nous de découvrir le travail de développement en équipe sur un projet de plus grande envergure que ce que nous avons réalisé jusqu'à maintenant.	