\section{Introduction}

Le jeu de rôles (JdR) est un jeu qui regroupe plusieurs joueurs dans la même salle qui incarnent chacun un personnage pour jouer ensemble. Ils sont guidé par un Maître de Jeu qui orchestre le bon déroulement des sessions. Le jeu est accompagné de beaucoup de matériels tels que les dés, le plateau et les jetons. Cependant, un inconvénient se fait ressentir : il faut que toutes les personnes se trouvent dans la même pièce. Il n'est donc pas toujours facile pour les joueurs de se retrouver, particulièrement lorsqu'ils se situent loin géographiquement. \\

Ayant ces éléments à l'esprit, nous avons développé une interface graphique qui permet d'organiser des parties de jeu de rôles à distance sur réseau. Nous avons fait attention à ne pas dépayser les utilisateurs en conservant au mieux l'atmosphère du jeu de plateau habituel. \\

Ce projet est une opportunité pour nous de découvrir le travail de développement en équipe sur un projet de plus grande envergure que ce que nous avons réalisé jusqu'à maintenant.	