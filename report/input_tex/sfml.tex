\section{SFML}

Au début de notre projet, nous avions pensé à utiliser une bibliothèque graphique spécialisée dans la réalisation de jeu, SFML (\textit{Simple and Fast Multimedia Library}).\\

Cette bibliothèque nous aurait permis de réaliser facilement des animations et toute sorte d'effet graphique. 
Cependant, en utilisant SFML, de mauvaises réactions ont été constatées, notamment au niveau de l'affichage de la fenetre SFML.
Après plusieurs difficultés à faire cohabiter Qt et SFML, nous avons, au bout de deux semaines, nous avons décidé de réaliser l'application exclusivement avec Qt. Pour cela, nous avons dû repenser entièrement le système de carte déjà réalisé avec SFML.