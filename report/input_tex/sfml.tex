\section{SFML}

Au début de notre projet, nous avions pensé à utiliser une bibliothèque graphique spécialisée dans la réalisation de jeu, SFML (\textit{Simple and Fast Multimedia Library}).\\

Cette bibliothèque nous aurait permis de réaliser facilement des animations et divers effets graphiques. 
Cependant, en utilisant SFML, de mauvaises réactions ont été constatées, notamment au niveau de l'affichage de la fenêtre SFML dans la MDIArea de Qt.
Après avoir tenté d'inclure SFML dans Qt durant deux semaines, nous avons décidé de réaliser l'application exclusivement avec Qt. En effet, la carte était fonctionnelle, mais nous redoutions l'apparition de problèmes futurs. Nous avons donc du repenser entièrement le système de carte déjà réalisé avec SFML.