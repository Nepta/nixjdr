\subsection{Git}
\subsubsection{Logiciel}
Git est un logiciel de gestion de code source et de version. Git capture l'état
dans lequel se trouvent les fichiers à chaque nouvelle version.
Il garde un historique de chaque version ajouter afin de pouvoir facilement revenir dessus, les comparer ou bien travailler sur une nouvelle version (branche) sans modifier celles existante.
De plus en partageant un dépôt commun, Git permet à plusieurs personne de travailler sur le même projet en résolvant quasi automatiquement les éventuelles conflit
(ou, à défaut, obliger les développeurs à les résoudre).

\subsubsection{GitHub}
Afin de pouvoir travailler à distance, nous utilisons le site \textit{GitHub} pour héberger notre dépôt.
Ce site en plus de fournir un accès via internet propose une multitude de graphique sur l'avancé du projet ainsi qu'un tableau de bord récapitulatif de celui-ci.
% TODO images illisibles, à mettre en annexe? à découper en plus petites images?

\begin{figure}[h!]
	\centering
	\includegraphics[width=0.5\textwidth]{img/state_git_graph.png}
	\caption{graphe des versions git}
\end{figure}

\newpage
\begin{figure}[h!]
	\centering
	\includegraphics[width=1.0\textwidth]{img/state_git_graph_2.png}
	\caption{graphe des versions git - suite}
	\label{fig:notification}
\end{figure}
